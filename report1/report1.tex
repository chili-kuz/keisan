\documentclass[a4paper,12pt]{jarticle}
\usepackage[dvipdfmx]{graphicx}
\usepackage{amsmath,amssymb}
\usepackage{subfigure}
\usepackage{comment}
\usepackage{array}
\usepackage{setspace} % setspaceパッケージのインクルード
\usepackage{ascmac}
%\usepackage{fancybx}

\setlength{\hoffset}{0cm}
\setlength{\oddsidemargin}{-3mm}
\setlength{\evensidemargin}{-3cm}
\setlength{\marginparsep}{0cm}
\setlength{\marginparwidth}{0cm}
\setlength{\textheight}{24.7cm}
\setlength{\textwidth}{17cm}
\setlength{\topmargin}{-45pt}


\renewcommand{\baselinestretch}{1.6}
\renewcommand{\floatpagefraction}{1}
\renewcommand{\topfraction}{1}
\renewcommand{\bottomfraction}{1}
\renewcommand{\textfraction}{0}
\renewcommand{\labelenumi}{(\roman{enumi})}
%\renewcommand{\figurename}{Fig.} %図をFig.にする
\renewcommand{\baselinestretch}{1}

%図のキャプションからコロン:を消す
\makeatletter
\long\def\@makecaption#1#2{% #1=図表番号、#2=キャプション本文
\sbox\@tempboxa{#1. #2}
\ifdim \wd\@tempboxa >\hsize
#1 #2\par 
\else
\hb@xt@\hsize{\hfil\box\@tempboxa\hfil}
\fi}
\makeatother

\begin{document}
%
\title{\vspace{-30mm} \fbox{\large{計算数学特論~レポート課題~~~機械知能工学専攻~~16344217~~津上~祐典}}}
\date{}
%
\maketitle
%
\vspace{-30mm}
%\parindent = 0pt %すべての段落で字下げをしない
%
%%%%%%%%%%%%%%%%%%%%%%%%%%%%%%%%%%%%%%%%%%%%%%%%%%%%%%%%%
\setlength{\abovedisplayskip}{0pt} % 数式上部のマージン
\setlength{\belowdisplayskip}{0pt} % 数式下部のマージン
%\setstretch{1.3} % ページ全体の行間を設定
%%%%%%%%%%%%%%%%%%%%%%%%%%%%%%%%%%%%%%%%%%%%%%%%%%%%%%%%%%

%%%%%%%%%%%%%%%%%%%%%%%%%%%%%%
\section*{課題1}
%%%%%%%%%%%%%%%%%%%%%%%%%%%%%%
\vspace{-3mm}
%%%%%%%%%%%%%%%%%%%%%%%%%%%%%%%
\subsection*{(1)~$(Z_p*,\times)$が群の構造を持つことを示せ.}
%%%%%%%%%%%%%%%%%%%%%%%%%%%%%%%%
%\vspace{-3mm}
%%%%%%%%%%%%%%%%%%%%%%%%%%%%%%%%
\subsubsection*{条件1.~演算$\times$が閉じている.}
%%%%%%%%%%%%%%%%%%%%%%%%%%%%%%%%
\vspace{-4mm}
演算$\times$が閉じている,つまり$a,b\in Z_p*$のとき,
$a\times b(\mathrm{mod}~p)$ $\in Z_p*$であることを示せば良い.
$Z_p*= \left\{1,2,\cdots,p-1 \right\}$より,$a\times b(\mathrm{mod}~p)\in Z_p*$
が成り立たない場合は,$a\times b(\mathrm{mod}~p)=0$のときである.背理法によ
り$a\times b(\mathrm{mod}~p)=0$が成り立たないことを示すことで演算$\times$が
閉じていることを示す.

はじめに,$a\times b(\mathrm{mod}~p)=0$と仮定する.
$a\times b(\mathrm{mod}~p)=0$となるのは
%
\begin{equation}
 a \times b =kp
\end{equation}
%
となるときである.ただし$k$は自然数である.式変形すると,
%
\begin{equation}
 \frac{a \times b}{p} = k
\end{equation}
%
となる.右辺の$k$は自然数であるから,左辺も自然数でないとならない.しか
し,左辺の分母$p$は素数である.また$a,b\in Z_p*$であり,$a\neq p,b\neq p$
であるから左辺は約分できず自然数とならない.よって,(2)式また(1)式は成り
立たない.これは最初の仮定に矛盾する.よって,$a\times b(\mathrm{mod}~p)=0$
は成り立たず,演算$\times$が閉じていると言える.
%
\vspace{-6mm}
%%%%%%%%%%%%%%%%%%%%%%%%%%%%%%%%
\subsubsection*{条件2.~結合法則が成立する.}
%%%%%%%%%%%%%%%%%%%%%%%%%%%%%%%%
\vspace{-4mm}
$a,b,c\in Z_p*$とすると,$a,b,c$は自然数であるから,結合法則
$a \times (b \times c)=(a \times b) \times c$
は成り立つ.
%
\vspace{-6mm}
%%%%%%%%%%%%%%%%%%%%%%%%%%%%%%%%
\subsubsection*{条件3.~単位元,逆元がある.}
%%%%%%%%%%%%%%%%%%%%%%%%%%%%%%%%
\vspace{-4mm}
$a\in Z_p*$とすると,$a \times 1 = 1 \times a = a$が成り立ち,単位元であ
る$1$が存在する.また,$ax+py=\mathrm{gcd}(a,p)=1$とすると,
%
\begin{eqnarray}
 ax + py = 1\\
 ax = 1(\mathrm{mod}~p) = 1
\end{eqnarray}
%
となり,$a$の逆元は$x$であり逆元は存在する.以上の3つの条件全て満たして
いるので,$(Z_p*,\times)$が群の構造を持つ.
\vspace{-5mm}
%%%%%%%%%%%%%%%%%%%%%%%%%%%%%%%
\subsection*{(2)~(1)を用いて,フェルマーの小定理$a^{p-1}(\mathrm{mod}~p)=1$($a$は$Z_p*$の任意の要素)を示せ.}
%%%%%%%%%%%%%%%%%%%%%%%%%%%%%%%%
%はじめに,$a,b,c\in Z_p*~,~b\neq c$のとき,$a \times b \neq a \times c$を
%背理法を用いて示す.$a,b,c\in Z_p*~,~b\neq c$のとき
%$a \times b = a\times c$と仮定する.両辺に左から$a$の逆元をかけると
%
%\begin{eqnarray}
 %a^{-1}(a \times b) &=& a^{-1} (a \times c) \\
 %b &=& c
%\end{eqnarray}
%
%となり,$b\neq c$と矛盾する.
%$a,b,c\in Z_p*~,~b\neq c$のとき,$a \times b \neq a \times c$は成り立つ.
$Z_p*$の要素をすべてかけたものを考える.
%
\begin{equation} \label{equ:12}
 1 \times 2 \times \cdots \times (p-1) 
\end{equation}
%
次に,$Z_p*$の要素に$a\in Z_p*$をかけて,そのすべてをかけたもの考える.
%
\begin{equation} \label{equ:1a}
 (1 \times a)\times (2 \times a)
  \times \cdots \times \left\{ (p-1) \times a\right\}
\end{equation}
%
ここで$a,b,c\in Z_p*~,~b\neq c$のとき,$a \times b \neq a \times c$より,
式の$(1 \times a),(2 \times a),\cdots,(p-1) \times a$はすべて異なる要素
である.また,課題1(1)より演算$\times$は閉じているので
(\ref{equ:12}),(\ref{equ:1a})式は等しい.
%
\begin{eqnarray}
 1 \times 2 \times \cdots \times (p-1) &=& (1 \times a)\times (2 \times a)
  \times \cdots \times \left\{ (p-1) \times a\right\} \\
&=&a^{p-1}(1 \times 2 \times \cdots \times (p-1))
\end{eqnarray}
%
となり,
%
\begin{equation}
 a^{p-1} (\mathrm{mod}~p)=1
\end{equation}
%
が成り立つ.
\vspace{-10mm}
%%%%%%%%%%%%%%%%%%%%%%%%%%%%%%
\section*{課題2}
%%%%%%%%%%%%%%%%%%%%%%%%%%%%%%
\vspace{-3mm}
%%%%%%%%%%%%%%%%%%%%%%%%%%%%%%%%%%%%%
\subsection*{(1)~$p=7,q=13$で公開鍵$e$と秘密鍵$d$を設定せよ.}
%%%%%%%%%%%%%%%%%%%%%%%%%%%%%%%%%%%%%
公開鍵$e$と秘密鍵$d$は
%
\begin{align*}
  n&=p \times q = 7 \times 13 = 91 \\
 \lambda &= \mathrm{lcm}(p-1,q-1) = \mathrm{lcm}(7-1,13-1) =\mathrm{lcm}(6,12)=12 \\
 1&=\mathrm{gcd}(d,\lambda) \iff \mathrm{gcd}(d,12)=1 ~~\therefore d=5\\
 ed &= 1(\mathrm{mod}~\lambda) \iff e5=1(\mathrm{mod}~12) ~~\therefore e=5
\end{align*}
%
より$e=5,~d=5$となる.
\vspace{-6mm}
%%%%%%%%%%%%%%%%%%%%%%%%%%%%%%%%%%%%%%%
\subsection*{(2)~コーディングを以下とする.a=1,$\cdots$,z=26,A=27,$\cdots$,Z=52,0=60,$\cdots$,9=69}
%%%%%%%%%%%%%%%%%%%%%%%%%%%%%%%%%%%%%%%
\vspace{-3mm}
%%%%%%%%%%%%%%%%%%%%%%%%%%%%%%%%
\subsubsection*{(a)~kitを送受信せよ.}
%%%%%%%%%%%%%%%%%%%%%%%%%%%%%%%%
\vspace{-4mm}
k,i,tの文字コードはそれぞれ$x_1=11,x_2=9,x_3=20$であるので暗号化コード
$c_1,c_2,c_3$は
%
\begin{align*}
 c_1&=11^5(\mathrm{mod}~91)=11^2\cdot11^2\cdot11(\mathrm{mod}~91)=30\cdot30\cdot11(\mathrm{mod}~91)=81\cdot11(\mathrm{mod}~91)=72\\
 c_2&=9^5(\mathrm{mod}~91)=9^3\cdot9^2(\mathrm{mod}~91)=1\cdot9^2(\mathrm{mod}~91)=81\\
 c_3&=20^5(\mathrm{mod}~91)=20^2\cdot20^2\cdot20(\mathrm{mod}~91)=36\cdot36\cdot20(\mathrm{mod}~91)=36\cdot83(\mathrm{mod}~91)=76
\end{align*}
%
となる.また復元化コードは
%
\begin{align*}
 X_1&=72^5(\mathrm{mod}~91)=72^2\cdot72^2\cdot72(\mathrm{mod}~91)=88\cdot88\cdot72(\mathrm{mod}~91)=9\cdot72(\mathrm{mod}~91)=11\\
 X_2&=81^5(\mathrm{mod}~91)=81^2\cdot81^2\cdot81(\mathrm{mod}~91)=9\cdot9\cdot81(\mathrm{mod}~91)=9\\
 X_3&=76^5(\mathrm{mod}~91)=76^2\cdot76^2\cdot76(\mathrm{mod}~91)=43\cdot43\cdot76(\mathrm{mod}~91)=29\cdot76(\mathrm{mod}~91)=20
\end{align*}
%
となり文字コードと一致する.
\vspace{-7mm}
%%%%%%%%%%%%%%%%%%%%%%%%%%%%%%%%
\subsubsection*{(b)~各人のイニシャル2文字を送受信せよ.}
%%%%%%%%%%%%%%%%%%%%%%%%%%%%%%%%
\vspace{-4mm}
私のイニシャルYTの文字コードは$x_1=51,x_2=46$であるので暗号化コード
$c_1,c_2$は
%
\begin{align*}
 c_1&=51^5(\mathrm{mod}~91)=51^2\cdot51^2\cdot51(\mathrm{mod}~91)=53\cdot53\cdot51(\mathrm{mod}~91)=79\cdot51(\mathrm{mod}~91)=25\\
 c_2&=46^5(\mathrm{mod}~91)=46^2\cdot46^2\cdot46(\mathrm{mod}~91)=23\cdot23\cdot46(\mathrm{mod}~91)=74\cdot46(\mathrm{mod}~91)=37
\end{align*}
%
となる.また復元化コードは
%
\begin{align*}
 X_1&=25^5(\mathrm{mod}~91)=25^2\cdot25^2\cdot25(\mathrm{mod}~91)=79\cdot79\cdot25(\mathrm{mod}~91)=53\cdot25(\mathrm{mod}~91)=51\\
 X_2&=37^5(\mathrm{mod}~91)=37^2\cdot37^2\cdot37(\mathrm{mod}~91)=4\cdot4\cdot37(\mathrm{mod}~91)=46
\end{align*}
%
となり文字コードと一致する.
\vspace{-7mm}
%%%%%%%%%%%%%%%%%%%%%%%%%%%%%%%%
\subsubsection*{(c)~学籍番号の下4桁を送受信せよ.}
%%%%%%%%%%%%%%%%%%%%%%%%%%%%%%%%
\vspace{-4mm}
文字コードは$x_1=64,x_2=62,x_3=61,x_4=67$であるので暗号化コード
$c_1,c_2,c_3,c_4$は
%
\begin{align*}
 c_1&=64^5(\mathrm{mod}~91)=64^2\cdot64^2\cdot64(\mathrm{mod}~91)=1\cdot1\cdot64(\mathrm{mod}~91)=64\\
 c_2&=62^5(\mathrm{mod}~91)=62^2\cdot62^2\cdot62(\mathrm{mod}~91)=22\cdot22\cdot62(\mathrm{mod}~91)=29\cdot62(\mathrm{mod}~91)=69\\
 c_3&=61^5(\mathrm{mod}~91)=61^2\cdot61^2\cdot61(\mathrm{mod}~91)=81\cdot81\cdot61(\mathrm{mod}~91)=9\cdot61(\mathrm{mod}~91)=3\\
 c_4&=67^5(\mathrm{mod}~91)=67^2\cdot67^2\cdot67(\mathrm{mod}~91)=30\cdot30\cdot67(\mathrm{mod}~91)=81\cdot67(\mathrm{mod}~91)=58
\end{align*}
%
となる.また復元化コードは
%
\begin{align*}
 X_1&=64^5(\mathrm{mod}~91)=64^2\cdot64^2\cdot64(\mathrm{mod}~91)=1\cdot1\cdot64(\mathrm{mod}~91)=64\\
 X_2&=69^5(\mathrm{mod}~91)=69^2\cdot69^2\cdot69(\mathrm{mod}~91)=29\cdot29\cdot69(\mathrm{mod}~91)=22\cdot69(\mathrm{mod}~91)=62\\
 X_3&=3^5(\mathrm{mod}~91)=61\\
 X_4&=58^5(\mathrm{mod}~91)=58^2\cdot58^2\cdot58(\mathrm{mod}~91)=88\cdot88\cdot58(\mathrm{mod}~91)=9\cdot58(\mathrm{mod}~91)=67
\end{align*}
%
となり文字コードと一致する.
\end{document}
