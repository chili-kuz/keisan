\documentclass[a4paper,12pt]{jarticle}
\usepackage[dvipdfmx]{graphicx}
\usepackage{amsmath,amssymb}
\usepackage{subfigure}
\usepackage{comment}
\usepackage{array}
\usepackage{setspace} % setspaceパッケージのインクルード
\usepackage{ascmac}
%\usepackage{fancybx}

\setlength{\hoffset}{0cm}
\setlength{\oddsidemargin}{-3mm}
\setlength{\evensidemargin}{-3cm}
\setlength{\marginparsep}{0cm}
\setlength{\marginparwidth}{0cm}
\setlength{\textheight}{24.7cm}
\setlength{\textwidth}{17cm}
\setlength{\topmargin}{-45pt}


\renewcommand{\baselinestretch}{1.6}
\renewcommand{\floatpagefraction}{1}
\renewcommand{\topfraction}{1}
\renewcommand{\bottomfraction}{1}
\renewcommand{\textfraction}{0}
\renewcommand{\labelenumi}{(\roman{enumi})}
%\renewcommand{\figurename}{Fig.} %図をFig.にする
\renewcommand{\baselinestretch}{1}

%図のキャプションからコロン:を消す
\makeatletter
\long\def\@makecaption#1#2{% #1=図表番号、#2=キャプション本文
\sbox\@tempboxa{#1. #2}
\ifdim \wd\@tempboxa >\hsize
#1 #2\par 
\else
\hb@xt@\hsize{\hfil\box\@tempboxa\hfil}
\fi}
\makeatother

\begin{document}
%
\title{\vspace{-30mm} \fbox{\large{計算数学特論~レポート課題3~~~機械知能工学専攻~~16344217~~津上~祐典}}}
\date{}
%
\maketitle
%
\vspace{-20mm}
%\parindent = 0pt %すべての段落で字下げをしない
%
%%%%%%%%%%%%%%%%%%%%%%%%%%%%%%%%%%%%%%%%%%%%%%%%%%%%%%%%%
\setlength{\abovedisplayskip}{0pt} % 数式上部のマージン
\setlength{\belowdisplayskip}{0pt} % 数式下部のマージン
%\setstretch{1.3} % ページ全体の行間を設定
%%%%%%%%%%%%%%%%%%%%%%%%%%%%%%%%%%%%%%%%%%%%%%%%%%%%%%%%%%

%%%%%%%%%%%%%%%%%%%%%%%%%%%%%%
%\section*{課題.~NIS}
%%%%%%%%%%%%%%%%%%%%%%%%%%%%%%
\vspace{-3mm}
%
 \begin{table}[htbp]
  \fontsize{10pt}{11pt}\selectfont
  \begin{tabular}{c|cccccccccccc}
       &[P,1]&[P,2]&[P,3]&[Q,1]&[Q,2]&[Q,3]&[R,1]&[R,2]&[R,3]&[S,1]&[S,2]&[S,3]\\\hline
   inf &\{4\}&\{2\}&\{1,5\}&\{5\}&\{3\}&\{4\}&\{\}&\{1\}&\{3,5\}&\{\}&\{\}&\{1,4\}\\
   sup &\{3,4\}&\{2,3\}&\{1,5\}&\{1,5\}&\{2,3\}&\{1,2,4\}&\{2\}&\{1,4\}&\{2,3,4,5\}&\{2,3,5\}&\{3,5\}&\{1,2,4\} 
  \end{tabular}
 \end{table}
%
%\vspace{-10mm}
以下に,$minsup,OUTACC,minacc,maxsup,INACC,maxacc$の計算結果を示す.レポー
ト枚数の都合上,計算途中は省略した.また,すべてのCertain~rule~,~Possible~rule
を求めた.
%%%%%%%%%%%%%%%%%%%%%%%%%%%%%%%%
\subsubsection*{(1)~$\tau_1:[P,1]=>[S,1]$}
%%%%%%%%%%%%%%%%%%%%%%%%%%%%%%%%
\vspace{-4mm}
%
\begin{align*}
 minsup(\tau_1)=0~,~maxsup(\tau_1)=0
\end{align*}
%
\vspace{-10mm}
%%%%%%%%%%%%%%%%%%%%%%%%%%%%%%%%
\subsubsection*{(2)~$\tau_2:[P,1]=>[S,2]$}
%%%%%%%%%%%%%%%%%%%%%%%%%%%%%%%%
\vspace{-4mm}
%
\begin{align*}
 minsup(\tau_2)=0~,~maxsup(\tau_2)=\frac{1}{5}\\
 INACC=\{3\}~,~maxacc(\tau_2)=\frac{1}{2}
\end{align*}
%
\vspace{-10mm}
%%%%%%%%%%%%%%%%%%%%%%%%%%%%%%%%
\subsubsection*{(3)~$\tau_3:[P,1]=>[S,3]$}
%%%%%%%%%%%%%%%%%%%%%%%%%%%%%%%%
\vspace{-4mm}
%
\begin{align*}
 minsup(\tau_3)=\frac{1}{5}~,~OUTACC=\{3\}~,~minacc(\tau_3)=\frac{1}{2}\\
 maxsup(\tau_3)=\frac{1}{5}~,~INACC=\{\}~,~maxacc(\tau_3)=1
\end{align*}
%
\vspace{-10mm}
%%%%%%%%%%%%%%%%%%%%%%%%%%%%%%%%
\subsubsection*{(4)~$\tau_4:[P,2]=>[S,1]$}
%%%%%%%%%%%%%%%%%%%%%%%%%%%%%%%%
\vspace{-4mm}
%
\begin{align*}
 minsup(\tau_4)=0~,~maxsup(\tau_4)=\frac{2}{5}~,~INACC=\{3\}~,~maxacc(\tau_4)=1
\end{align*}
%
\vspace{-10mm}
%%%%%%%%%%%%%%%%%%%%%%%%%%%%%%%%
\subsubsection*{(5)~$\tau_5:[P,2]=>[S,2]$}
%%%%%%%%%%%%%%%%%%%%%%%%%%%%%%%%
\vspace{-4mm}
%
\begin{align*}
 minsup(\tau_5)=0~,~maxsup(\tau_5)=\frac{1}{5}~,~INACC=\{3\}~,~maxacc(\tau_5)=\frac{1}{2}
\end{align*}
%
\vspace{-10mm}
%%%%%%%%%%%%%%%%%%%%%%%%%%%%%%%%
\subsubsection*{(6)~$\tau_6:[P,2]=>[S,3]$}
%%%%%%%%%%%%%%%%%%%%%%%%%%%%%%%%
\vspace{-4mm}
%
\begin{align*}
 minsup(\tau_6)=0~,~maxsup(\tau_6)=\frac{1}{5}~,~INACC=\{\}~,~maxacc(\tau_6)=1
\end{align*}
%
\vspace{-10mm}
%%%%%%%%%%%%%%%%%%%%%%%%%%%%%%%%
\subsubsection*{(7)~$\tau_7:[P,3]=>[S,1]$}
%%%%%%%%%%%%%%%%%%%%%%%%%%%%%%%%
\vspace{-4mm}
%
\begin{align*}
 minsup(\tau_7)=0~,~maxsup(\tau_7)=\frac{1}{5}~,~INACC=\{\}~,~maxacc(\tau_7)=\frac{1}{2}
\end{align*}
%
\vspace{-10mm}
%%%%%%%%%%%%%%%%%%%%%%%%%%%%%%%%
\subsubsection*{(8)~$\tau_8:[P,3]=>[S,2]$}
%%%%%%%%%%%%%%%%%%%%%%%%%%%%%%%%
\vspace{-4mm}
%
\begin{align*}
 minsup(\tau_8)=0~,~maxsup(\tau_8)=\frac{1}{5}~,~INACC=\{\}~,~maxacc(\tau_8)=\frac{1}{2}
\end{align*}
%
\vspace{-10mm}
%%%%%%%%%%%%%%%%%%%%%%%%%%%%%%%%
\subsubsection*{(9)~$\tau_9:[P,3]=>[S,3]$}
%%%%%%%%%%%%%%%%%%%%%%%%%%%%%%%%
\vspace{-4mm}
%
\begin{align*}
 minsup(\tau_9)=\frac{1}{5}~,~OUTACC=\{\}~,~minacc(\tau_9)=\frac{1}{2}\\
 maxsup(\tau_9)=\frac{1}{5}~,~INACC=\{\}~,~maxacc(\tau_9)=\frac{1}{2}
\end{align*}
%
\vspace{-10mm}
%%%%%%%%%%%%%%%%%%%%%%%%%%%%%%%%
\subsubsection*{(10)~$\tau_{10}:[Q,1]=>[S,1]$}
%%%%%%%%%%%%%%%%%%%%%%%%%%%%%%%%
\vspace{-4mm}
%
\begin{align*}
 minsup(\tau_{10})=0~,~maxsup(\tau_{10})=\frac{1}{5}~,~INACC=\{\}~,~maxacc(\tau_{10})=1
\end{align*}
%
\vspace{-10mm}
%%%%%%%%%%%%%%%%%%%%%%%%%%%%%%%%
\subsubsection*{(11)~$\tau_{11}:[Q,1]=>[S,2]$}
%%%%%%%%%%%%%%%%%%%%%%%%%%%%%%%%
\vspace{-4mm}
%
\begin{align*}
 minsup(\tau_{11})=0~,~maxsup(\tau_{11})=\frac{1}{5}~,~INACC=\{\}~,~maxacc(\tau_{11})=1
\end{align*}
%
\vspace{-10mm}
%%%%%%%%%%%%%%%%%%%%%%%%%%%%%%%%
\subsubsection*{(12)~$\tau_{12}:[Q,1]=>[S,3]$}
%%%%%%%%%%%%%%%%%%%%%%%%%%%%%%%%
\vspace{-4mm}
%
\begin{align*}
 minsup(\tau_{12})=0~,~maxsup(\tau_{12})=\frac{1}{5}~,~INACC=\{1\}~,~maxacc(\tau_{12})=\frac{1}{2}
\end{align*}
%
\vspace{-10mm}
%%%%%%%%%%%%%%%%%%%%%%%%%%%%%%%%
\subsubsection*{(13)~$\tau_{13}:[Q,2]=>[S,1]$}
%%%%%%%%%%%%%%%%%%%%%%%%%%%%%%%%
\vspace{-4mm}
%
\begin{align*}
 minsup(\tau_{13})=0~,~maxsup(\tau_{13})=\frac{2}{5}~,~INACC=\{2\}~,~maxacc(\tau_{13})=1
\end{align*}
%
\vspace{-10mm}
%%%%%%%%%%%%%%%%%%%%%%%%%%%%%%%%
\subsubsection*{(14)~$\tau_{14}:[Q,2]=>[S,2]$}
%%%%%%%%%%%%%%%%%%%%%%%%%%%%%%%%
\vspace{-4mm}
%
\begin{align*}
 minsup(\tau_{14})=0~,~maxsup(\tau_{14})=\frac{1}{5}~,~INACC=\{\}~,~maxacc(\tau_{14})=1
\end{align*}
%
\vspace{-10mm}
%%%%%%%%%%%%%%%%%%%%%%%%%%%%%%%%
\subsubsection*{(15)~$\tau_{15}:[Q,2]=>[S,3]$}
%%%%%%%%%%%%%%%%%%%%%%%%%%%%%%%%
\vspace{-4mm}
%
\begin{align*}
 minsup(\tau_{15})=0~,~maxsup(\tau_{15})=\frac{1}{5}~,~INACC=\{2\}~,~maxacc(\tau_{15})=\frac{1}{2}
\end{align*}
%
\vspace{-10mm}
%%%%%%%%%%%%%%%%%%%%%%%%%%%%%%%%
\subsubsection*{(16)~$\tau_{16}:[Q,3]=>[S,1]$}
%%%%%%%%%%%%%%%%%%%%%%%%%%%%%%%%
\vspace{-4mm}
%
\begin{align*}
 minsup(\tau_{16})=0~,~maxsup(\tau_{16})=\frac{1}{5}~,~INACC=\{2\}~,~maxacc(\tau_{16})=\frac{1}{2}
\end{align*}
%
\vspace{-10mm}
%%%%%%%%%%%%%%%%%%%%%%%%%%%%%%%%
\subsubsection*{(17)~$\tau_{17}:[Q,3]=>[S,2]$}
%%%%%%%%%%%%%%%%%%%%%%%%%%%%%%%%
\vspace{-4mm}
%
\begin{align*}
 minsup(\tau_{17})=0~,~maxsup(\tau_{17})=0
\end{align*}
%
\vspace{-10mm}
%%%%%%%%%%%%%%%%%%%%%%%%%%%%%%%%
\subsubsection*{(18)~$\tau_{18}:[Q,3]=>[S,3]$}
%%%%%%%%%%%%%%%%%%%%%%%%%%%%%%%%
\vspace{-4mm}
%
\begin{align*}
 minsup(\tau_{18})=\frac{1}{5}~,~OUTACC=\{2\}~,~minacc(\tau_{18})=\frac{1}{2}\\
 maxsup(\tau_{18})=\frac{3}{5}~,~INACC=\{1,2\}~,~maxacc(\tau_{18})=1
\end{align*}
%
\vspace{-10mm}
%%%%%%%%%%%%%%%%%%%%%%%%%%%%%%%%
\subsubsection*{(19)~$\tau_{19}:[R,1]=>[S,1]$}
%%%%%%%%%%%%%%%%%%%%%%%%%%%%%%%%
\vspace{-4mm}
%
\begin{align*}
 minsup(\tau_{19})=0~,~maxsup(\tau_{19})=\frac{1}{5}~,~INACC=\{2\}~,~maxacc(\tau_{19})=1
\end{align*}
%
\vspace{-10mm}
%%%%%%%%%%%%%%%%%%%%%%%%%%%%%%%%
\subsubsection*{(20)~$\tau_{20}:[R,1]=>[S,2]$}
%%%%%%%%%%%%%%%%%%%%%%%%%%%%%%%%
\vspace{-4mm}
%
\begin{align*}
 minsup(\tau_{20})=0~,~maxsup(\tau_{20})=0
\end{align*}
%
\vspace{-10mm}
%%%%%%%%%%%%%%%%%%%%%%%%%%%%%%%%
\subsubsection*{(21)~$\tau_{21}:[R,1]=>[S,3]$}
%%%%%%%%%%%%%%%%%%%%%%%%%%%%%%%%
\vspace{-4mm}
%
\begin{align*}
 minsup(\tau_{21})=0~,~maxsup(\tau_{21})=\frac{1}{5}~,~INACC=\{2\}~,~maxacc(\tau_{21})=1
\end{align*}
%
\vspace{-10mm}
%%%%%%%%%%%%%%%%%%%%%%%%%%%%%%%%
\subsubsection*{(22)~$\tau_{22}:[R,2]=>[S,1]$}
%%%%%%%%%%%%%%%%%%%%%%%%%%%%%%%%
\vspace{-4mm}
%
\begin{align*}
 minsup(\tau_{22})=0~,~maxsup(\tau_{22})=0
\end{align*}
%
\vspace{-10mm}
%%%%%%%%%%%%%%%%%%%%%%%%%%%%%%%%
\subsubsection*{(23)~$\tau_{23}:[R,2]=>[S,2]$}
%%%%%%%%%%%%%%%%%%%%%%%%%%%%%%%%
\vspace{-4mm}
%
\begin{align*}
 minsup(\tau_{23})=0~,~maxsup(\tau_{23})=0
\end{align*}
%
\vspace{-10mm}
%%%%%%%%%%%%%%%%%%%%%%%%%%%%%%%%
\subsubsection*{(24)~$\tau_{24}:[R,2]=>[S,3]$}
%%%%%%%%%%%%%%%%%%%%%%%%%%%%%%%%
\vspace{-4mm}
%
\begin{align*}
 minsup(\tau_{24})=\frac{1}{5}~,~OUTACC=\{\}~,~maxsup(\tau_{24})=1
\end{align*}
%
\vspace{-10mm}
%%%%%%%%%%%%%%%%%%%%%%%%%%%%%%%%
\subsubsection*{(25)~$\tau_{25}:[R,3]=>[S,1]$}
%%%%%%%%%%%%%%%%%%%%%%%%%%%%%%%%
\vspace{-4mm}
%
\begin{align*}
 minsup(\tau_{25})=0~,~maxsup(\tau_{25})=\frac{3}{5}~,~INACC=\{2\}~,~maxacc(\tau_{25})=1
\end{align*}
%
\vspace{-10mm}
%%%%%%%%%%%%%%%%%%%%%%%%%%%%%%%%
\subsubsection*{(26)~$\tau_{26}:[R,3]=>[S,2]$}
%%%%%%%%%%%%%%%%%%%%%%%%%%%%%%%%
\vspace{-4mm}
%
\begin{align*}
 minsup(\tau_{26})=0~,~maxsup(\tau_{26})=\frac{2}{5}~,~INACC=\{\}~,~maxacc(\tau_{26})=1
\end{align*}
%
\vspace{-10mm}
%%%%%%%%%%%%%%%%%%%%%%%%%%%%%%%%
\subsubsection*{(27)~$\tau_{27}:[R,3]=>[S,3]$}
%%%%%%%%%%%%%%%%%%%%%%%%%%%%%%%%
\vspace{-4mm}
%
\begin{align*}
 minsup(\tau_{27})=0~,~maxsup(\tau_{27})=\frac{2}{5}~,~INACC=\{2,4\}~,~maxacc(\tau_{27})=\frac{1}{2}
\end{align*}
%

\vspace{10mm}
以上まででminsupはOKでminaccはNOなのは
%
\begin{align*}
 [P,1]=>[S,3]~,~[P,3]=>[S,3]~,~[Q,3]=>[S,3]
\end{align*}
%
の3パターンである.これらの条件部に条件を追加する.
%%%%%%%%%%%%%%%%%%%%%%%%%%%%%%%%
\subsubsection*{(28)~$\tau_{28}:[P,1]\land[Q,3]=>[S,3]$}
%%%%%%%%%%%%%%%%%%%%%%%%%%%%%%%%
%\vspace{-4mm}
$inf([P,1]\land[Q,3])=\{4\}~,~sup([P,1]\land[Q,3])=\{4\}$より
%
\begin{align*}
 minsup(\tau_{28})=\frac{1}{5}~,~OUTACC=\{\}~,~minacc(\tau_{28})=1
\end{align*}
%
となる.

以上より,$support\geq0.2~,~accuracy\geq0.8$を満たすPossible~ruleは,
%
\begin{align*}
 &[P,1]=>[S,3]~,~[P,2]=>[S,1]~,~[P,2]=>[S,3]~,~[Q,1]=>[S,1]~,~[Q,1]=>[S,2] \\
 &[Q,2]=>[S,1]~,~[Q,2]=>[S,2]~,~[Q,3]=>[S,3]~,~[R,1]=>[S,1]~,~[R,1]=>[S,3] \\
 &[R,2]=>[S,3]~,~[R,3]=>[S,1]~,~[R,3]=>[S,2]
\end{align*}
%
であり,Certain~ruleは,
%
\begin{align*}
 [R,2]=>[S,3]~,~[P,1]\land[Q,3]=>[S,3]
\end{align*}
%
である.
\end{document}
